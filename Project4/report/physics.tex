A simple physical system we studied is two electrons confined in a three dimensional H.O. potential. 
It's Hamiltonian (in atomic unit (a.u.)) has three parts
\begin{equation}
	\hat{H}=\sum_{i=1,2}(-\frac{1}{2}\nabla_i^2+\frac{1}{2}\omega^2r_i^2)+\frac{1}{r_{12}},
\end{equation}
where two in parenthesis are their kinetic and potential energies, third is the Coulomb repulsion between electrons. 
The first trail wave function we used is 
\begin{equation}\label{eq:t1}
	\Phi_{t1}(\vec{r}_1,\vec{r}_2)=C\rm{exp}(-\alpha \omega (\vec{r}_1+\vec{r}_2)/2)
\end{equation}
which simply the product of two electrons' wave functions in H.O. potential without repulsion.
The $C$ works for normalization and $\alpha$ is the only one variational parameter here.

One step further, in order to get rid of divergence when $r_{12}\rightarrow 0$, we have second trail wave function
\begin{equation}\label{eq:t2}
	\Phi_{t2}(\vec{r}_1,\vec{r}_2)=C\rm{exp}(-\alpha \omega (\vec{r}_1+\vec{r}_2)/2)\rm{exp}(\frac{r_{12}}{2(1+\beta r_{12})})
\end{equation}
which satisfies the cusp condition.
In this $\Phi_{t2}$ we add an extra term called Jastrow factor.
And now, it has two variational parameters $\alpha$ and $\beta$.
Substituting these two trail wave functions to \ref{local}, we can obtain their local energies
\begin{equation}
	E_{L1}= \frac{1}{2}\omega^2(\vec{r}_1^2+\vec{r}_2^2)(1-\alpha ^2)+3 \alpha \omega + \frac{1}{r_{12}}
\end{equation}
\begin{equation}
	E_{L2}= E_{L1}+\frac{1}{2(1+\beta r_{12}^2)}\left( \alpha \omega r_{12}-r_{12}^2-\frac{2}{r_{12}}+\frac{2\beta}{1+\beta r_{12}}\right).
\end{equation}
We can see the second local energy is much complicated than the first one with the help of the Jastrow factor. 
We will see whether it worthwhile to pay this extra price from our results.
