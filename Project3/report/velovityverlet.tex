The Velocity-Verlet method is widely used in molecular dynamics calculation as it overcomes above defects. It conserves energy within small round-off errors\cite{toxvaerd2012energy}.

Staring from the Taylor expansions under same discretization for both displacement and velocity

\begin{equation}
	\label{eqt::verlet}
	\begin{aligned}
		x_{i+1} = x_i + x_i^{(1)}h + x_i^{(2)}h^2 + O(h^3),\\
		v_x^{i+1} = v_x^{i} + v_x^{i(1)}h + v_x^{i(2)}h^2 + O(h^3),
	\end{aligned}
\end{equation}
with initial values $x_0$ and $v_x^0$ at time $t_0$. 
We truncate at the third term and evaluate $v_x^{i(2)}h \approx v_x^{i+1(1)}-v_x^{i(1)}$, 
from which and Eq. \ref{eqt::verlet} we can see that the Velocity-Verlet method is a two-step method with a local error $O(h^3)$

In the Earth-Sun system, with this method, we can formulate Eq. \ref{earthsunodes} to

\begin{equation}
	\left\{  
             \begin{array}{lr}  
             	x_{i+1} = x_i + v_x^{i}h - \frac{4\pi^2x_i}{r_i^3}\frac{h^2}{2} \\
				y_{i+1} = y_i + v_y^{i}h - \frac{4\pi^2y_i}{r_i^3}\frac{h^2}{2} \\
             	v_x^{i+1} = v_x^{i} - (\frac{4\pi^2x_i}{r_i^3}+\frac{4\pi^2x_{i+1}}{r_{i+1}^3})\frac{h}{2}\\
             	v_y^{i+1} = v_y^{i} - (\frac{4\pi^2y_i}{r_i^3}+\frac{4\pi^2y_{i+1}}{r_{i+1}^3})\frac{h}{2}.
			\end{array}  
	\right.	
\end{equation} 
We show our realization of the Velocity-Verlet method in Algorithm \ref{alg::verlet}. 
Compared with the Euler's forward method, we can see the processes in this method are more complicated with more calculations involved.

\begin{algorithm}[tb]
	\caption{The Velocity-Verlet method for the Earth-Sun system. It initials from a circular orbit.}
	\label{alg::verlet}
	\KwIn{$x_0=1$, $y_0=0$, $v_{x}^0=0$, $v_{y}^0=2\pi$, $M_{\mathrm{E}}$, $M_{\mathrm{S}}$, $G$ }
	\KwOut{$\vec{x}$=($x_0$,$x_1$,...,$x_N$), $\vec{y}$, $\vec{v}_x$, $\vec{v}_y$, $\vec{E}_k$, $\vec{E}_p$, $\vec{E}$, $\vec{L}_z$} 
	$r$ = sqrt($x_0^2$+$y_0^2$)\;
	$a_x^0$ = $\frac{4\pi^2x_{i-1}}{r_{i-1}^3}$;
    $a_y^0$ = $\frac{4\pi^2y_{i-1}}{r_{i-1}^3}$\;
	// $i$ is deferent time points $t_i=t_0+ih$\;
    \For{$i=1;i<=N;i++$}
    {$x_{i} = x_{i-1} + v_x^{i-1}h - \frac{a_x^0h^2}{2}$;
    $y_{i} = y_{i-1} + v_y^{i-1}h - \frac{a_y^0h^2}{2}$\;
    $r$ = sqrt($x_i^2$+$y_i^2$)\;
    $a_x^1$ = $\frac{4\pi^2x_{i}}{r_{i}^3}$;
    $a_y^1$ = $\frac{4\pi^2y_{i}}{r_{i}^3}$\;
    $v_x^{i} = v_x^{i-1} - \frac{(a_x^0+a_x^1)h}{2}h$;
    $v_y^{i} = v_y^{i-1} - \frac{(a_y^0+a_y^1)h}{2}h$\;
	$a_x^0$ = $a_x^1$;
    $a_y^0$ = $a_x^1$\;
    // Kinetic energy $E_k$, Potential energy $E_k$, Total energy $E$, Angular momentum in $\hat{z}$ $L_z$\;
    $E_k^i$ = 0.5$M_{\mathrm{E}}$($(v_x^{i})^2$+$(v_y^{i})^2$)\;
    $E_p^i$ = $-\frac{GM_{\mathrm{E}}}{r}$;
    $E^i$ = $E_p^i$ + $E_k^i$\;
    $L_z^i$ = $M_{\mathrm{E}}(x_iv_y^{i}-y_iv_x^{i})$\;
    }
	return $\vec{x}$, $\vec{y}$, $\vec{v}_x$, $\vec{v}_y$, $\vec{E}_k$, $\vec{E}_p$, $\vec{E}$, $\vec{L}_z$\;
\end{algorithm}
