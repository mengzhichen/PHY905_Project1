Historically, the discrepancy between the observed and calculated values of perihelion precession of Mercury were not able to be accounted by classical Newtonian mechanics.
Until the introduction of general relativity, the problem was solved.
It was also labeled as a great success of general relativity.
The value results from relativistic correction is about $\delta \varphi\approx 43^{''}$ per century\cite{wiki:xxx}.
In order to get this value, we introduce a correction to our gravitational force, so that the force becomes
\begin{equation}
	\vec{F}_G = \frac{GM_{\mathrm{S}}M_{\mathrm{M}}\hat{r}}{r^2}\left[1+\frac{3l^2}{r^2c^2}\right],
\end{equation}
where $M_{\mathrm{M}}$ is the mass of Mercury, $l$ is $\vec{r}\times \vec{v}$ and $c$ is the speed of light in vacuum.

In our calculation, we apply the Mercury mass and new force to calculate the acceleration.
The Sun-Mercury system initials from $x_0=0.3075$, $y_0=0$, $v_{x}^0=0$ and $v_{y}^0=12.44$. 
For this elliptical orbit, theoretical calculation gives a period $T_m\approx 0.24073$ yr.
We also notice for a resolution higher to $1^{''}$, we have to use a very small step size.

Before setting up the step size, we did a rough estimation.
The Mercury moves about 420 circles in a century which is about 5.5E08 degrees. 
To distinguish $1^{''}$ within a century movement, we need at least a step size $h$ = 100/5.5E08 $\approx$ 1.84E-07. 
Therefore, $h$ = 1.0E-07 would be a good choice.

We try different $h$ in our calculations and the final results are given in Table \ref{tab::mercury}.
The outcomes justify our estimation that the error will less than $1^{''}$ for $h$ = 1.0E-07.
Eventually, our calculation gives $\delta \varphi=42.9333^{''}$ with $h$ smaller to 5.0E-08 which is very close to the theoretical value 43$^{''}$.
\begin{table}[tb]
	\centering
	\caption{The perihelion precession angle $\delta \varphi$ of Mercury due to relativistic correction after one century. }
	\label{my-label}
	\label{tab::mercury}
	\begin{tabular}{cc}
	\hline
	\hline
	$h$ (yr)         & $\delta \varphi$     \\
	\hline
	5.00E-07 & 53.2277$^{''}$ \\
	2.00E-07 & 45.8665$^{''}$ \\
	1.00E-07 & 42.8183$^{''}$ \\
	5.00E-08 & 42.9333$^{''}$\\
	\hline
	\hline
	\end{tabular}
\end{table}