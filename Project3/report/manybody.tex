In the above Sec. \ref{earthsun}, we simplify the calculation of the orbit of the Earth by taking into account only it's interaction with the Sun. 
This simplification is reasonable as the Sun is much heavier than other planets in the solar system. 

However, if we want to go a step further to get a more precise description of the Earth's orbit. 
We have to include distortions from other seven planets as well as the Pluto.
We should also abandon our previous static Sun approximation but using the real center of mass of the solar system.
Until now, we have a new set of ODEs for the Earth

\begin{equation}
	\left\{  
             \begin{array}{lr}  
             	\frac{dx}{dt} = v_x \\
				\frac{dy}{dt} = v_y \\
            	\frac{dv_x}{dt} = \sum\limits_{i=1}^{n}-\frac{GM_{\mathrm{i}}x_i}{r_i^3} \\
				\frac{dv_y}{dt} = \sum\limits_{i=1}^{n}-\frac{GM_{\mathrm{i}}y_i}{r_i^3},
			\end{array}  
	\right. 	
\end{equation}
where $i$ runs over all other celestial bodies except the Earth itself. 
Besides the Earth, we have similar sets of ODEs for every celestial body.

By solving these coupled ODEs, we can obtain a full description for the solar system.
